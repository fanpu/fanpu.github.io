% You should title the file with a .tex extension (hw1.tex, for example)
\documentclass[11pt]{article}
% \usepackage{tikz}
\usepackage{amsmath}
\usepackage{blindtext}
\usepackage{multicol}
\usepackage{listings} %For code in appendix
\lstset{
    basicstyle=\ttfamily,
    mathescape
}
\usepackage{wasysym}
\usepackage{amsthm}
\usepackage{amssymb}
\usepackage{graphicx}
\usepackage{pdfpages}
\usepackage{float}
\usepackage{fancyhdr}
\usepackage[ruled,vlined]{algorithm2e}
\usepackage{xcolor}
\usepackage{mathtools}
\usepackage{amstext} % for \text macro
\usepackage{array}   % for \newcolumntype macro
% \usepackage{algorithm}
% \usepackage{algpseudocode}
% \usepackage[full]{complexity}
\newcolumntype{L}{>{$}l<{$}} % math-mode version of "l" column type
\DeclarePairedDelimiter{\ceil}{\lceil}{\rceil}

% \usepackage{hyperref}
% \hypersetup{
%     colorlinks=true,
%     linkcolor=blue,
%     filecolor=magenta,
%     urlcolor=cyan,
% }
\usepackage{CSTheoryToolkitCMUStyle}
\newcommand{\bigzero}{\mbox{\normalfont\Large\bfseries 0}}
\newcommand{\bigone}{\mbox{\normalfont\Large\bfseries 1}}
\newcommand{\mM}{\mathbf{M}}
\newcommand{\mA}{\mathbf{A}}
\newcommand{\mS}{\mathbf{S}}
\newcommand{\rvline}{\hspace*{-\arraycolsep}\vline\hspace*{-\arraycolsep}}

\definecolor{codegreen}{rgb}{0,0.6,0}
\definecolor{codegray}{rgb}{0.5,0.5,0.5}
\definecolor{codepurple}{rgb}{0.58,0,0.82}
\definecolor{backcolour}{rgb}{0.95,0.95,0.92}

\lstdefinestyle{mystyle}{
    backgroundcolor=\color{backcolour},
    commentstyle=\color{codegreen},
    keywordstyle=\color{magenta},
    numberstyle=\tiny\color{codegray},
    stringstyle=\color{codepurple},
    basicstyle=\ttfamily\footnotesize,
    breakatwhitespace=false,
    breaklines=true,
    captionpos=b,
    keepspaces=true,
    numbers=left,
    numbersep=5pt,
    showspaces=false,
    showstringspaces=false,
    showtabs=false,
    tabsize=2
}

\lstset{style=mystyle}

\oddsidemargin0cm
\topmargin-2cm     %I recommend adding these three lines to increase the
\textwidth16.5cm   %amount of usable space on the page (and save trees)
\textheight23.5cm

\newcommand{\question}[2] {\vspace{.25in} \hrule\vspace{0.5em}
    \noindent{\bf #1: #2} \vspace{0.5em}
    \hrule \vspace{.10in}}
\renewcommand{\part}[1] {\vspace{.10in} {\bf (#1)}}

\newcommand{\myname}{Fan Pu Zeng}
\newcommand{\myandrew}{fzeng@andrew.cmu.edu}
\newcommand{\myhwnum}{1}

\setlength{\parindent}{0pt} \setlength{\parskip}{5pt plus 1pt}

\pagestyle{fancyplain}
\lhead{\fancyplain{}{\textbf{HW\myhwnum}}}      % Note the different brackets!
\rhead{\fancyplain{}{\myname\\ \myandrew}}
\chead{\fancyplain{}{Scratch Pad}}

\newcommand{\clearthin}{\usefont{\encodingdefault}{ClearSans-TLF}{thin}{n}}
\newcommand{\definequantifier}[3]{%1 = command, #2 = h or v, #3 = letter
    \if #2h%
        \DeclareRobustCommand{#1}{\scalebox{-1}[1]{\text{\clearthin#3}}}%
    \else
        \DeclareRobustCommand{#1}{\raisebox{\depth}{\scalebox{1}[-1]{\text{\clearthin#3}}}}%
    \fi
}

\DeclareMathOperator*{\argmax}{arg\,max}
\DeclareMathOperator*{\argmin}{arg\,min}

\allowdisplaybreaks

% Add rulers for algorithm2e
\RestyleAlgo{ruled}
\SetKwComment{Comment}{/* }{ */}

\newcommand{\nnz}[1]{\mathtt{nnz}(#1)}


\begin{document}

% \includepdf[pages=-]{hw10_handwritten.pdf}

\medskip                        % Skip a "medium" amount of space
% (latex determines what medium is)
% Also try: \bigskip, \littleskip

\thispagestyle{plain}
\begin{center}                  % Center the following lines
    {\Large 15-859 Algorithms for Big Data Assignment \myhwnum} \\
    \myname \\
    \myandrew \\
\end{center}

\question{1}{Scratcy Scratch}

\begin{align*}
    T = \left\{ (x, \sin \frac{1}{x}) : x \in (0, 1] \right\} \cup \left\{ \left( 0, 0 \right) \right\}
\end{align*}

\begin{align*}
    T = \{(x, \sin \frac{1}{x} ) : x \in (0, 1] \} \cup \{ ( 0, 0 ) \}                                               \\
    T = \left\{ \left( x, \sin \frac{1}{x} \right) : x \in (0, 1] \right\} \cup \left\{ \left( 0, 0 \right) \right\} \\
    T = \braces*{ \parens*{ x, \sin \frac{1}{x} } : x \in (0, 1] } \cup \braces*{ \parens*{ 0, 0 }}                  \\
\end{align*}




\begin{align*}
    \left( \frac{1}{x} \right)           \\
    \left[ \frac{1}{x} \right]           \\
    \left\{ \frac{1}{x} \right\}         \\
    \left\| \frac{1}{x} \right\|         \\
    \lvert \frac{1}{x} \rvert            \\
    \left\| \frac{1}{x} \right\|         \\
    \left\lceil \frac{1}{x} \right\rceil \\
\end{align*}

\newpage

\begin{figure}
    \includegraphics{example-image-b}
    \caption{Landing/Login Page Example}
    \label{fig:env-gg}
\end{figure}
\begin{figure}
    \includegraphics{example-image-a}
    \caption{Landing/Login Page Example}
    \label{fig:env-perf}
\end{figure}

\begin{multicols}{2}
    We denote the historical trajectory as
    ${ \tau=\left(s_1, a_1, \ldots, a_{t-1}, s_t\right) }$
    and action-observation history $(\mathrm{AOH})$ for
    player $i$ as
    ${ \tau^i=\left(\Omega^i\left(s_1\right), a_1, \ldots, a_{t-1}, \Omega^i\left(s_t\right)\right) }$,
    which encodes the trajectory from player $i$ 's point of view.

    We evaluated the policy periodically during training by testing it without exploration noise.
    Figure~\ref{fig:env-perf} shows the performance curve for a selection of environments. We also report
    results with components of our algorithm.

    \blindtext\blindtext
    \begin{align*}
        X & = \left( X_1, ..., X_n \right)   & \text{(using ``...'')}                 & \\
        X & = \left( X_1, \dots, X_n \right) & \text{(using ``\textbackslash dots'')} &
    \end{align*}
\end{multicols}

``Hello World!''

\textit{Problem: Show that if $(x_n)_n$ converges to $x$ in the usual sense, then
    $\lim_{n \to \infty} x_n = \lim_{\mathcal{F}} x_n$.}

Suppose that $(x_n)_n$ converges to $x$. We show that this $x$ is also the
$\mathcal{F}$-limit of $(x_n)_n$.

\begin{proof}
    Take any $\varepsilon$. Then we know that for some large enough $N$, if $n \geq N$, then
    $x_n \in B_\varepsilon(x)$. Since every non-principal ultrafilter on $\N$ contains
    $\mathcal{F}_\infty$, then $\mathcal{F}$ also contains $ \left\{ n : n \geq N \right\} $,
    since the complement is finite. Therefore since filters are closed upwards, any
    sequence items $x_n$ with $n < N$ that happen to fall in the ball around $x$, i.e, $x_n \in B_\varepsilon(x)$
    is also contained in some filter element, so $\left\{  n \in \N : \lvert x_n - x \rvert < \varepsilon \right\} \in \mathcal{F}$, as desired.

\end{proof}

\begin{proof}
    Take any $\varepsilon$. Then we know that for some large enough $N$, if $n \geq N$, then
    $x_n \in B_\varepsilon(x)$. Since every non-principal ultrafilter on $\N$ contains
    $\mathcal{F}_\infty$, then $\mathcal{F}$ also contains $ \left\{ n : n \geq N \right\} $,
    since the complement is finite. Therefore since filters are closed upwards, any
    sequence items $x_n$ with $n < N$ that happen to fall in the ball around $x$,
    i.e, $x_n \in B_\varepsilon(x)$
    is also contained in some filter element, so
    $\left\{  n \in \N : \lvert x_n - x \rvert < \varepsilon \right\} \in \mathcal{F}$,
    as desired. \qedhere

    % Extra newline here!
\end{proof}

$$P(X) = \int P(X \mid z; \theta) P(z) dz.$$

$$P(X) = \int xyz dx.$$

\section{A}

% \begin{definition}
%     A sequence $(x_n)_n : \mathbbm{N} \to X$ is a Cauchy sequence if 
%     $\forall \epsilon > 0, \exists \, N \in \mathbbm{N}$ such that 
%     $\forall \, n, m \geq N, \, d \left( x_n, x_m \right) < \epsilon$.
% \end{definition}

% \begin{definition}[Using \textbackslash mathbb]
%     A sequence $(x_n)_n : \mathbb{N} \to X$ is a Cauchy sequence if 
%     ${\forall \varepsilon > 0, \exists \, N \in \mathbb{N}}$ such that 
%     ${\forall \, n, m \geq N, \, d \left( x_n, x_m \right) < \varepsilon}$.
% \end{definition}

% \begin{definition}[Using \textbackslash mathbbm]
%     A sequence $(x_n)_n : \mathbbm{N} \to X$ is a Cauchy sequence if 
%     ${\forall \varepsilon > 0, \exists \, N \in \mathbbm{N}}$ such that 
%     ${\forall \, n, m \geq N, \, d \left( x_n, x_m \right) < \varepsilon}$.
% \end{definition}

\begin{definition}[Using *]
    A distribution on matrices $\bS \in \R^{k \times n}$ has the $\left( \varepsilon, \delta, \ell \right)$-JL
    moment property if for all $x \in \R^n$ with $\lvert x \rvert_2 = 1$,
    $$ \Ex_{\bS} \left\lvert \left\lvert \bS x \right\rvert_2^2 - 1 \right\rvert^\ell \leq \varepsilon^\ell * \delta. $$
\end{definition}

\begin{definition}[Using \textbackslash cdot]
    A distribution on matrices $\bS \in \R^{k \times n}$ has the $\left( \varepsilon, \delta, \ell \right)$-JL
    moment property if for all $x \in \R^n$ with $\lvert x \rvert_2 = 1$,
    $$ \Ex_{\bS} \left\lvert \left\lvert \bS x \right\rvert_2^2 - 1 \right\rvert^\ell \leq \varepsilon^\ell \cdot \delta. $$
\end{definition}

$ \mathbf{Pr} \left[ X \geq a \right] \leq \frac{\mathbf{E}[X]}{a}$

$ x\Gr_{X} \left[ X \geq a \right] \leq \frac{\mathbf{E}[X]}{a}$

$ x\Hr_{X} \left[ X \geq a \right] \leq \frac{\mathbf{E}[X]}{a}$

$ \Ex_{X \sim} \left[ X \geq a \right] \leq \frac{\mathbf{E}[X]}{a}$


$p(\bz, \bx) = p(\bz) p(\bx \mid \bz)$

$p(\bz, \bx) = p(\bz) p(\bx | \bz)$

$KL \left( P \; \middle\| \; Q \right)$

$KL \left( P \; \middle\| \; Q \right)$


We evaluated the policy periodically during training by testing it without exploration noise.
Figure~\ref{fig:env-perf} shows the performance curve for a selection of environments.

$$\sum_{i=1}^n$$

\begin{align*}
    \nabla_{\mu} (\mathbb{E}_{x\sim q_{\mu}} f(x)) & = \nabla_{\mu} \int_x f(x) q_{\mu}(x) dx                                     \\
                                                   & =  \int_x f(x) (\nabla_{\mu} \log q_{\mu}(x))  q_{\mu}(x) dx                 \\
                                                   & = \mathbb{E}_{x \sim q_{\mu}} \left(f(x) \nabla_{\mu} \log q_{\mu}(x)\right)
\end{align*}

\begin{align*}
    \nabla_{\mu} (\mathbb{E}_{x\sim q_{\mu}} f(x))  = & \nabla_{\mu} \int_x f(x) q_{\mu}(x) dx                                     \\
    =                                                 & \int_x f(x) (\nabla_{\mu} \log q_{\mu}(x))  q_{\mu}(x) dx                  \\
    =                                                 & \mathbb{E}_{x \sim q_{\mu}} \left(f(x) \nabla_{\mu} \log q_{\mu}(x)\right)
\end{align*}

$ \Pr \left[ X \geq a \right] \leq \frac{\Ex[X]}{a}$

$$\phi, \Phi, \varphi$$ ,

\begin{theorem}[Gibbs Variational Principle] \label{thm:gibbs}
    Let $p(\mathbf{z}, \mathbf{x})$ be a joint distribution over latent variables and observables. Then,
    $$p(\mathbf{z} \mid \mathbf{x}) = \argmax_{q(\mathbf{z} \mid \mathbf{x}) : \mbox{ distribution over } \mathbf{z}}$$
\end{theorem}

\renewcommand{\bA}{\boldsymbol{A}}
$$\min_x \lvert \bA x - b \rvert_2^2$$

$$\boldsymbol{M}$$

\newcommand{\dotprod}[2]{\langle #1, #2 \rangle}
\renewcommand{\bu}{\boldsymbol{u}}
\renewcommand{\bv}{\boldsymbol{v}}
$$\left\lvert \dotprod{\bu}{\bv} \right\rvert^2 \leq \dotprod{\bu}{\bu} \cdot \dotprod{\bv}{\bv}$$


\begin{align*}
    p(\mbox{Disease} \mid \mbox{Symptoms})
     & = \frac{p(\mbox{Disease}, \mbox{Symptoms})}{p(\mbox{Symptoms})} \\
     & = \frac{p(\mbox{Disease}, \mbox{Symptoms})}{\sum_{\mbox{Disease}^\prime \in \mbox{Diseases}}p(\mbox{Disease}^\prime, \mbox{Symptoms})} \\
\end{align*}

\begin{align*}
    \frac{1}{\mathcal{Z}(\theta)}\exp \left( \sum_{\substack{i, j \in [n] \\ i \neq j}} x_i x_j \theta_{ij} + \sum_{i \in [n]} x_i \theta_i \right),
\end{align*}

\begin{align*}
    \mathcal{Z}(\theta) = \sum_{\bx \in \left\{ \pm 1 \right\}^n} p(\theta, \bx)
\end{align*}

Let $p(\bz, \bx)$ be a joint distribution ovre latent variables $\bz$ and
observables $\bx$. Then
$$ p(\bz \mid \bx) = \argmax_{ q(\bz \mid \bx) : \text{distribution over } \bz } H(q(\bz \mid \bx)) + \E_{\bz \sim q()} $$


    <a href="">
\end{document}
