% You should title the file with a .tex extension (hw1.tex, for example)
\documentclass[11pt]{article}
% \usepackage{tikz}
\usepackage{amsmath}
\usepackage{listings} %For code in appendix
\lstset{
    basicstyle=\ttfamily,
    mathescape
}
\usepackage{wasysym}
\usepackage{amsthm}
\usepackage{amssymb}
\usepackage{graphicx}
\usepackage{pdfpages}
\usepackage{float}
\usepackage{fancyhdr}
\usepackage[ruled,vlined]{algorithm2e}
\usepackage{xcolor}
\usepackage{mathtools}
\usepackage{amstext} % for \text macro
\usepackage{array}   % for \newcolumntype macro
% \usepackage{algorithm}
% \usepackage{algpseudocode}
% \usepackage[full]{complexity}
\newcolumntype{L}{>{$}l<{$}} % math-mode version of "l" column type
\DeclarePairedDelimiter{\ceil}{\lceil}{\rceil}

% \usepackage{hyperref}
% \hypersetup{
%     colorlinks=true,
%     linkcolor=blue,
%     filecolor=magenta,
%     urlcolor=cyan,
% }
\usepackage{CSTheoryToolkitCMUStyle}
\newcommand{\bigzero}{\mbox{\normalfont\Large\bfseries 0}}
\newcommand{\bigone}{\mbox{\normalfont\Large\bfseries 1}}
\newcommand{\mM}{\mathbf{M}}
\newcommand{\mA}{\mathbf{A}}
\newcommand{\mS}{\mathbf{S}}
\newcommand{\rvline}{\hspace*{-\arraycolsep}\vline\hspace*{-\arraycolsep}}

\definecolor{codegreen}{rgb}{0,0.6,0}
\definecolor{codegray}{rgb}{0.5,0.5,0.5}
\definecolor{codepurple}{rgb}{0.58,0,0.82}
\definecolor{backcolour}{rgb}{0.95,0.95,0.92}

\lstdefinestyle{mystyle}{
    backgroundcolor=\color{backcolour},
    commentstyle=\color{codegreen},
    keywordstyle=\color{magenta},
    numberstyle=\tiny\color{codegray},
    stringstyle=\color{codepurple},
    basicstyle=\ttfamily\footnotesize,
    breakatwhitespace=false,
    breaklines=true,
    captionpos=b,
    keepspaces=true,
    numbers=left,
    numbersep=5pt,
    showspaces=false,
    showstringspaces=false,
    showtabs=false,
    tabsize=2
}

\lstset{style=mystyle}

\oddsidemargin0cm
\topmargin-2cm     %I recommend adding these three lines to increase the
\textwidth16.5cm   %amount of usable space on the page (and save trees)
\textheight23.5cm

\newcommand{\question}[2] {\vspace{.25in} \hrule\vspace{0.5em}
    \noindent{\bf #1: #2} \vspace{0.5em}
    \hrule \vspace{.10in}}
\renewcommand{\part}[1] {\vspace{.10in} {\bf (#1)}}

\newcommand{\myname}{Fan Pu Zeng}
\newcommand{\myandrew}{fzeng@andrew.cmu.edu}
\newcommand{\myhwnum}{1}

\setlength{\parindent}{0pt} \setlength{\parskip}{5pt plus 1pt}

\pagestyle{fancyplain}
\lhead{\fancyplain{}{\textbf{HW\myhwnum}}}      % Note the different brackets!
\rhead{\fancyplain{}{\myname\\ \myandrew}}
\chead{\fancyplain{}{Scratch Pad}}

\newcommand{\clearthin}{\usefont{\encodingdefault}{ClearSans-TLF}{thin}{n}}
\newcommand{\definequantifier}[3]{%1 = command, #2 = h or v, #3 = letter
    \if #2h%
        \DeclareRobustCommand{#1}{\scalebox{-1}[1]{\text{\clearthin#3}}}%
    \else
        \DeclareRobustCommand{#1}{\raisebox{\depth}{\scalebox{1}[-1]{\text{\clearthin#3}}}}%
    \fi
}

\DeclareMathOperator*{\argmax}{arg\,max}
\DeclareMathOperator*{\argmin}{arg\,min}

\allowdisplaybreaks

% Add rulers for algorithm2e
\RestyleAlgo{ruled}
\SetKwComment{Comment}{/* }{ */}

\newcommand{\nnz}[1]{\mathtt{nnz}(#1)}


\begin{document}

% \includepdf[pages=-]{hw10_handwritten.pdf}

\medskip                        % Skip a "medium" amount of space
% (latex determines what medium is)
% Also try: \bigskip, \littleskip

\thispagestyle{plain}
\begin{center}                  % Center the following lines
    {\Large 15-859 Algorithms for Big Data Assignment \myhwnum} \\
    \myname \\
    \myandrew \\
\end{center}

\question{1}{Scratcy Scratch}

\begin{align*}
    \nabla_\theta \E_{\tau \sim P_\theta(\tau)} [R(\tau)]
     & = \nabla_\theta \sum\limits_\tau P_\theta(\tau) R(\tau)                \\
     & = \sum\limits_\tau \nabla_\theta  P_\theta(\tau) R(\tau) \tag{uh oh..} \\
\end{align*}

\begin{align*}
    \sum\limits_\tau \nabla_\theta  P_\theta(\tau) R(\tau)
     & = \sum\limits_\tau \frac{ P_\theta(\tau) }{ P_\theta(\tau) } \nabla_\theta  P_\theta(\tau) R(\tau)  \\
     & = \sum\limits_\tau P_\theta(\tau) \frac{ \nabla_\theta  P_\theta(\tau)  }{ P_\theta(\tau) } R(\tau) \\
     & = \sum\limits_\tau P_\theta(\tau) \nabla_\theta  \log  P_\theta(\tau) R(\tau)                       \\
     & = \E_{\tau \sim P_\theta(\tau)} \left[ \nabla_\theta  \log  P_\theta(\tau) R(\tau)  \right]         \\
     & \approx \frac{1}{N} \sum\limits_{i=1}^N \nabla_\theta  \log  P_\theta(\tau_i) R(\tau_i)             \\
\end{align*}

\begin{align*}
    \nabla_\theta  \log  P_\theta(\tau)
     & = \nabla_\theta  \log  \left[
        \prod_{t=0}^H P(s_{t+1} \mid s_t, a_t) \cdot \pi_\theta (a_t \mid s_t),
    \right]                          \\
     & = \nabla_\theta  \left[
        \sum\limits_{t=0}^H \log P(s_{t+1} \mid s_t, a_t) + \log \pi_\theta (a_t \mid s_t)
        \right] \\
     & = \nabla_\theta \sum\limits_{t=0}^H \log \pi_\theta (a_t \mid s_t) \tag{first term does not depend on $\theta$, becomes zero} \\
     & = \sum\limits_{t=0}^H \nabla_\theta \log \pi_\theta (a_t \mid s_t) \\
\end{align*}

\end{document}